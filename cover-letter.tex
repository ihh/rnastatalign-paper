\documentclass[10pt]{letter}
\pagestyle{plain}
%\usepackage[ansinew]{inputenc} 

%\addtolength{\topmargin}{-0.5in}

%\sloppy
%\nonfrenchspacing

\renewcommand{\baselinestretch}{1.0}

\address{
Ian Holmes\\
Department of Bioengineering\\
University of California\\
Berkeley, CA 94720\\
}

\signature{}
\begin{document}
\begin{letter}{}

\opening{Dear Editors:}

\vspace{\baselineskip}

The new science of {\bf Ancestral Sequence Reconstruction} is bringing evolutionary origins within experimental reach.
This field has come of age in the past five years.
Many ancient protein sequences have been reconstructed and experimentally tested
(Benner {\em et al}, Nature 2003;
Thornton {\em et al}, Science 2007;
Liberles (ed.), Oxford University Press 2007).
There have also been major attempts at the
computational reconstruction of genomes
(Blanchette, Green, Miller \& Haussler, Genome Research 2004;
Paten {\em et al}, Genome Research, in press).

Here, we present initial steps towards a theory, tool-set and benchmark for phylogenetic reconstruction of structured RNA,
such as ribosomal RNA.
This year is the fortieth anniversary of the {\bf RNA world hypothesis}
(Woese, Harper \& Row 1968).
It's also the 40th anniversary of the all-RNA primordial ribosome
(Crick, JMB 1968),
a structural model of which was recently proposed
(Smith, Gutell {\em et al}, Biology Direct, 2008).
Yet to our knowledge, the attached manuscript represents the first broad pilot study of concrete algorithms for reconstructing RNA.

Our paper describes several new software tools, all backed by simulation and/or benchmarking results.
However, we feel that the key contribution is a complete, self-consistent theoretical framework for modeling (and reconstructing) RNA evolution on a phylogenetic tree.
This theory is the RNA analogue of the {\bf transducer theory} for proteins and DNA
(Bradley and Holmes, Bioinformatics 2007),
which was recently used in the first whole-genome mammalian-ancestor reconstruction
(Paten, Herrero, Birney, Holmes \& Flicek, Genome Research, in press).
The theory is closely related to that of ``Statistical Alignment''
(Wong, Suchard and Huelsenbeck, Science 2008)
and sequence automata for reconstructing indel histories
(L\"{o}ytynoja \& Goldman, Science 2008).
Unlike those previous theories (which model only indels and point substitutions),
our model includes covariant base-pair substitutions and changes in RNA structure.

\newpage
All authors have read and approved of this manuscript.

%As editor, we suggest David Haussler.

%We suggest the following reviewers:
%David Haussler (if not editor),
%Jeff Thorne,
%Marc Suchard,
%Jotun Hein,
%Sean Eddy.
%We have provided contact details for these reviewers in the online
%submission process.

\closing{Sincerely,\vskip 20pt Robert K. Bradley\\Ian Holmes}

%% If you have things to enclose, specify here
%\encl{Encl. 1}
%\encl{Encl. 2}

\end{letter}

\end{document}
