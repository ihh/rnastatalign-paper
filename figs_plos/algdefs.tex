
% amsmath package, useful for mathematical formulas
\usepackage{amsmath}
% amssymb package, useful for mathematical symbols
\usepackage{amssymb}

% graphicx package, useful for including eps and pdf graphics
% include graphics with the command \includegraphics
\usepackage{graphicx}

% cite package, to clean up citations in the main text. Do not remove.
\usepackage{cite}

\usepackage{color} 

% Use doublespacing - comment out for single spacing
%\usepackage{setspace} 
%\doublespacing


% Text layout
\topmargin 0.0cm
\oddsidemargin 0.5cm
\evensidemargin 0.5cm
\textwidth 16cm 
\textheight 21cm

% Bold the 'Figure #' in the caption and separate it with a period
% Captions will be left justified
\usepackage[labelfont=bf,labelsep=period,justification=raggedright]{caption}

% Use the PLoS provided bibtex style
\bibliographystyle{PLoS2009}

% Remove brackets from numbering in List of References
\makeatletter
\renewcommand{\@biblabel}[1]{\quad#1.}
\makeatother


% Leave date blank
\date{}

\pagestyle{empty}
%\pagestyle{myheadings}
%% ** EDIT HERE **
%% Please insert a running head of 30 characters or less.  
%% Include it twice, once between each set of braces
%\markboth{Modeling structured RNA}{Modeling structured RNA}

%% ** EDIT HERE **
%% PLEASE INCLUDE ALL MACROS BELOW

\usepackage{setspace}
\doublespacing



\newcommand\titlestring{Evolutionary triplet models of structured RNA}
\newcommand\authorstring{
Robert K. Bradley$^{1}$, 
Ian Holmes$^{1,2,\ast}$
\\
\textbf{1} Biophysics Graduate Group, University of California, Berkeley, CA, USA
\\
\textbf{2} Department of Bioengineering, University of California, Berkeley, CA, USA
\\
$\ast$ E-mail: indiegram@postbox.biowiki.org
}

\usepackage{array}

\usepackage[boxed]{algorithm2e}
\usepackage{mathrsfs}

% Labels & references for sections, figures and tables
% Commented out \secref definition & replaced \seclabel with a dummy, because PLoS doesn't like numbered section references -- IH, 11/11/08
% (This edit appears to have been reversed as of 5/29/09 - presumably so that the supplementary information compiles OK using these same definitions - IH, 5/29/09)
\newcommand{\secref}[1]{Section~\ref{sec:#1}}
\newcommand{\seclabel}[1]{\label{sec:#1}}

\newcommand{\secname}[1]{``#1''}  % PLoS-style section names

% "Text S1", "Text S2", etc.
\newcommand{\supptext}[1]{Text S#1}

% "Dataset S1", "Dataset S2", etc.
\newcommand{\dataset}[1]{Dataset S#1}

\newcommand{\appref}[1]{Appendix~\ref{app:#1}}
\newcommand{\applabel}[1]{\label{app:#1}}

\newcommand{\figref}[1]{Figure~\ref{fig:#1}}
\newcommand{\figlabel}[1]{\label{fig:#1}}

\newcommand{\tabnum}[1]{\ref{tab:#1}}
\newcommand{\tabref}[1]{Table~\tabnum{#1}}
\newcommand{\tablabel}[1]{\label{tab:#1}}

% Changed algorithms to external figures - IH, 7/17/09
%\newcommand{\algref}[1]{Algorithm~\ref{alg:#1}}
%\newcommand{\alglabel}[1]{\label{alg:#1}}
%\newcommand{\algref}[1]{Figure~\ref{alg:#1}}
%\newcommand{\alglabel}[1]{\label{alg:#1}}
\newcommand{\algref}[1]{}
\newcommand{\alglabel}[1]{}

\newcommand{\eqnref}[1]{Equation~\ref{eqn:#1}}
\newcommand{\eqnlabel}[1]{\label{eqn:#1}}

% Frequently used proper nouns that should have consistent typeface, capitalization, etc.
\newcommand{\pfold}{PFOLD}
\newcommand\darturl{{\tt http://biowiki.org/dart}}
\newcommand\indiegramurl{{\tt http://biowiki.org/IndieGram}}
\newcommand{\dart}{DART}

\newcommand{\bralibaseII}{BRalibaseII}
\newcommand{\balibase}{BAliBASE}

\newcommand{\stemloc}{Stemloc}
\newcommand{\stemlocama}{Stemloc-AMA}
\newcommand{\xrate}{XRate}
\newcommand{\evoldoer}{Evoldoer}
\newcommand{\evolsayer}{{\tt evolsayer.pl}}
\newcommand{\animateevolsayer}{{\tt animate-evolsayer.pl}}
\newcommand{\indiegram}{Indiegram}
\newcommand{\handel}{Handel}

\newcommand{\contrafold}{CONTRAFOLD}
\newcommand{\consan}{CONSAN}
\newcommand{\clustalw}{ClustalW}
\newcommand{\foldalign}{Foldalign}
\newcommand{\foldalignm}{FoldalignM}
\newcommand{\mastr}{MASTR}
\newcommand{\rnasampler}{RNASampler}
\newcommand{\murlet}{Murlet}
\newcommand{\ortheus}{Ortheus}
\newcommand{\rnaalifold}{RNAAlifold}
\newcommand{\viennarna}{ViennaRNA}

\newcommand{\RFAM}{RFAM}

\newcommand{\denovo}{{\em de novo}}

\newcommand{\nth}{\mathrm{n}^{\mathrm{th}}}
\newcommand{\mth}{\mathrm{m}^{\mathrm{th}}}

\newcommand{\type}[0]{\operatorname{type}}
\newcommand{\emit}[0]{\operatorname{emit}}
\newcommand{\absorb}[0]{\operatorname{absorb}}
\newcommand{\argmax}[0]{\operatorname{argmax}}
\newcommand{\argmin}[0]{\operatorname{argmin}}
\newcommand{\parent}[0]{\operatorname{parent}}

\newcommand{\weight}[0]{\operatorname{Weight}}
\newcommand{\transweight}[3]{t\left(#2,#3|\theta^{(#1)}\right)}  % {m}{from}{to}
\newcommand{\emissionweight}[3]{e_{#2}\left(#3|\theta^{(#1)}\right)}  % {m}{emitstate}{emission}
\newcommand{\totalemissionweight}[2]{\mathbb{e}_{#1}(#2)}

%\newcommand{\cyk}[7]{\gamma_{#1}(#2,#3,#4,#5,#6,#7)}
\newcommand{\cyk}[7]{\gamma_{#1}\left(#2,#3,#4\right)}
%\newcommand{\cyk}[7]{\gamma_{#1}(#2,#3,#4)}
%\newcommand{\inside}[7]{\alpha_{#1}(#2,#3,#4,#5,#6,#7)}
\newcommand{\inside}[7]{\alpha_{#1}\left(#2,#3,#4\right)}
%\newcommand{\outside}[7]{\beta_{#1}(#2,#3,#4,#5,#6,#7)}
\newcommand{\outside}[7]{\beta_{#1}\left(#2,#3\right,#4\right)}

\newcommand{\Tnull}{\mathrm{null}}
\newcommand{\Tgap}{\mathrm{GAP}}

\newcommand{\statetype}[1]{\mathtt{#1}}
\newcommand{\state}[3]{{}^{#2}\mathtt{#1}^{#3}}
% Note that this will only work for left-bifurcations!
%\newcommand{\bifurc}[3]{{\phantom{]}}^{#2}\mathtt{B[#3\,#1]}}}
\newcommand{\bifurc}[3]{\mathtt{#1[#2\,#3]}}

% state types for branch transducers
\newcommand{\Sstart}[0]{\statetype{Start}}
\newcommand{\Smatch}[0]{\statetype{Match}}
\newcommand{\Sinsert}[0]{\statetype{Insert}}
\newcommand{\Swait}[0]{\statetype{Wait}}
\newcommand{\Send}[0]{\statetype{End}}

% state types for evolutionary model
\newcommand{\Snull}[0]{\statetype{Null}}
\newcommand{\Sbifurc}[0]{\statetype{Bifurcation}}
\newcommand{\Semit}[0]{\statetype{Emit}}

\newcommand{\statevec}[1]{\left(\begin{array}{c}#1_1\\\vdots\\#1_N\end{array}\right)}
\newcommand{\fourvec}[4]{\left(\begin{array}{c}#1\\#2\\#3\\#4\end{array}\right)}
\newcommand{\nakedfourvec}[4]{\begin{array}{c}#1\\#2\\#3\\#4\end{array}}
\newcommand{\threevec}[3]{\left(\begin{array}{c}#1\\#2\\#3\end{array}\right)}
\newcommand{\nakedthreevec}[3]{\begin{array}{c}#1\\#2\\#3\end{array}}
\newcommand{\bvec}[1]{\boldsymbol{#1}}
\newcommand{\activenode}[1]{n(\bvec{#1})}

\newcommand{\foldenv}{\mathscr{F}}

\newcommand{\mDn}[0]{m \vartriangleright n}
\newcommand{\mNDn}[0]{m \ntriangleright n}
\newcommand{\mDEn}[0]{m \trianglerighteq n}
\newcommand{\mNDEn}[0]{m \ntrianglerighteq n}

\newcommand{\beqn}{\begin{equation}}
\newcommand{\eeqn}{\end{equation}}

\newcommand\gnons{{\cal N}}
\newcommand\gnulls{{\bf N}}
\newcommand\gbifs{{\bf B}}
\newcommand\gemits{{\bf E}}
\newcommand\gterms{\Omega}
\newcommand\grules{{\bf R}}
\newcommand\grhs{( \gnons \cup \Omega )^\ast}
\newcommand\gtrees{{\cal T}}

\newcommand\leftside[1]{{\cal L}(#1)}
\newcommand\rightside[1]{{\cal R}(#1)}
\newcommand\allrules[1]{{\cal A}(#1)}

%% END MACROS SECTION
